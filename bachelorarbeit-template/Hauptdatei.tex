\documentclass[
    12pt, % Schriftgröße
    DIV10,
    ngerman, % für Umlaute, Silbentrennung etc.
    a4paper, % Papierformat
    oneside, % einseitiges Dokument
    titlepage, % es wird eine Titelseite verwendet
    parskip=half, % Abstand zwischen Absätzen (halbe Zeile)
    headings=normal, % Größe der Überschriften verkleinern
    listof=totoc, % Verzeichnisse im Inhaltsverzeichnis aufführen
    bibliography=totoc, % Literaturverzeichnis im Inhaltsverzeichnis aufführen
    index=totoc, % Index im Inhaltsverzeichnis aufführen
    captions=tableheading, % Beschriftung von Tabellen unterhalb ausgeben
    final % Status des Dokuments (final/draft)
]{scrartcl}
\usepackage{geometry}
\geometry{a4paper,top=20mm,left=30mm,right=20mm,bottom=20mm}

\usepackage[latin1]{inputenc}
\usepackage[german]{babel}
\usepackage{tikz}
\usepackage{amsmath}
\usepackage{graphicx}
\usepackage{pgfplotstable} 
\usepackage{hyperref} % References as hyperlinks
\usepackage[noabbrev]{cleveref}
\usepackage{xr}
\usepackage[onehalfspacing]{setspace} % Zeilenabstand

\usepackage[
    automark, % Kapitelangaben in Kopfzeile automatisch erstellen
    headsepline, % Trennlinie unter Kopfzeile
    ilines % Trennlinie linksbündig ausrichten
]{scrpage2}

\usetikzlibrary {positioning, calc, arrows}
\definecolor {processblue}{cmyk}{0.96,0,0,0}
% Kopfzeile
\ihead{\headmark} % links
\chead{}
\setlength{\headheight}{15mm} % Höhe der Kopfzeile
\setheadsepline[text]{0.4pt} % Trennlinie unter Kopfzeile

% Fußzeile
%\ifoot{} % links
\cfoot{} % mitte
\ofoot{\pagemark} % rechts

\pagestyle{scrheadings}


\begin{document}
% hier werden die Trennvorschläge inkludiert
\input{latex_einstellungen/trennung}

%Schriftart Helvetica
%\changefont{phv}{m}{n}

% Titelseite %
\include{latex_einstellungen/deckblatt}

% römische Numerierung
%\pagenumbering{arabic}

% 1.5 facher Zeilenabstand
\onehalfspacing

% Sperrvermerk
\input{sperrvermerk}

% Einleitung / Abstract
\include{abstract}

% einfacher Zeilenabstand
\singlespacing

% Inhaltsverzeichnis anzeigen
\newpage
\tableofcontents

% das Abbildungsverzeichnis
%\newpage
% Abbildungsverzeichnis soll im Inhaltsverzeichnis auftauchen
\addcontentsline{toc}{section}{Abbildungsverzeichnis}
% Abbildungsverzeichnis endgueltig anzeigen
\listoffigures

% das Tabellenverzeichnis
%\newpage
% Abbildungsverzeichnis soll im Inhaltsverzeichnis auftauchen
\addcontentsline{toc}{section}{Tabellenverzeichnis}
% \fancyhead[L]{Abbildungsverzeichnis / Abkürzungsverzeichnis} %Kopfzeile links
% Abbildungsverzeichnis endgueltig anzeigen
\listoftables

%% WORKAROUND für Listings
%\makeatletter% --> De-TeX-FAQ
%\renewcommand*{\lstlistoflistings}{%
%  \begingroup
%    \if@twocolumn
%      \@restonecoltrue\onecolumn
%    \else
%      \@restonecolfalse
%    \fi
%    \lol@heading
%    \setlength{\parskip}{\z@}%
%    \setlength{\parindent}{\z@}%
%    \setlength{\parfillskip}{\z@ \@plus 1fil}%
%    \@starttoc{lol}%
%    \if@restonecol\twocolumn\fi
%  \endgroup
%}
%\makeatother% --> \makeatletter
% das Listingverzeichnis
%\newpage
% Listingverzeichnis soll im Inhaltsverzeichnis auftauchen
\addcontentsline{toc}{section}{Listingverzeichnis}
\fancyhead[L]{Abbildungs- / Tabellen- / Listingverzeichnis} %Kopfzeile links
\renewcommand{\lstlistlistingname}{Listingverzeichnis}
\lstlistoflistings
%%%%

% das Abkürzungsverzeichnis
%\newpage
% Abkürzungsverzeichnis soll im Inhaltsverzeichnis auftauchen
\addcontentsline{toc}{section}{Abkürzungsverzeichnis}
% das Abkürzungsverzeichnis entgültige Ausgeben
\fancyhead[L]{Abkürzungsverzeichnis} %Kopfzeile links
\input{latex_einstellungen/abkuezungen/abkuerzungen}
\printnomenclature

% Definiert Stegbreite bei zweispaltigem Layout
\setlength{\columnsep}{25pt}

%%%%%%% EINLEITUNG %%%%%%%%%%%%
%\twocolumn
\newpage
\fancyhead[L]{\nouppercase{\leftmark}} %Kopfzeile links

% 1,5 facher Zeilenabstand
\onehalfspacing

% einzelne Kapitel
\input{1_einleitung}

\input{2_kap1}

\input{3_kap2}

%....

\input{7_ausblick}

\input{8_fazit}

\onecolumn
% einfacher Zeilenabstand
\singlespacing
% Literaturliste soll im Inhaltsverzeichnis auftauchen
\newpage
\addcontentsline{toc}{section}{Literaturverzeichnis}
% Literaturverzeichnis anzeigen
\renewcommand\refname{Literaturverzeichnis}
\bibliography{Hauptdatei}

%% Index soll Stichwortverzeichnis heissen
% \newpage
% % Stichwortverzeichnis soll im Inhaltsverzeichnis auftauchen
% \addcontentsline{toc}{section}{Stichwortverzeichnis}
% \renewcommand{\indexname}{Stichwortverzeichnis}
% % Stichwortverzeichnis endgueltig anzeigen
% \printindex

\onehalfspacing
% evtl. Anhang
\newpage
\addcontentsline{toc}{section}{Anhang}
\fancyhead[L]{Anhang} %Kopfzeile links
\input{anhang/anhang}

% Eidesstattliche Erklärung
\addcontentsline{toc}{section}{Eidesstattliche Erklärung}
\include{erklaerung}

% leere Abschlussseite
\newpage
\thispagestyle{empty} % erzeugt Seite ohne Kopf- / Fusszeile
\section*{ }

\end{document}