\documentclass[
    12pt, % Schriftgr��e
    DIV10,
    ngerman, % f�r Umlaute, Silbentrennung etc.
    a4paper, % Papierformat
    oneside, % einseitiges Dokument
    titlepage, % es wird eine Titelseite verwendet
    parskip=half, % Abstand zwischen Abs�tzen (halbe Zeile)
    headings=normal, % Gr��e der �berschriften verkleinern
    listof=totoc, % Verzeichnisse im Inhaltsverzeichnis auff�hren
    bibliography=totoc, % Literaturverzeichnis im Inhaltsverzeichnis auff�hren
    index=totoc, % Index im Inhaltsverzeichnis auff�hren
    captions=tableheading, % Beschriftung von Tabellen unterhalb ausgeben
    final % Status des Dokuments (final/draft)
]{scrartcl}
\usepackage{geometry}
\geometry{a4paper,top=30mm,left=30mm,right=20mm,bottom=32mm}

\usepackage[latin1]{inputenc}
\usepackage[german]{babel}
\usepackage{tikz}
\usepackage{amsmath}
\usepackage{graphicx}
\usepackage{pgfplotstable} 
\usepackage{hyperref} % References as hyperlinks
\usepackage[noabbrev]{cleveref}
\usepackage{xr}

\usepackage[
    automark, % Kapitelangaben in Kopfzeile automatisch erstellen
    headsepline, % Trennlinie unter Kopfzeile
    ilines % Trennlinie linksb�ndig ausrichten
]{scrpage2}

\usetikzlibrary {positioning, calc, arrows}
\definecolor {processblue}{cmyk}{0.96,0,0,0}
% Kopfzeile
\ihead{\headmark} % links
\chead{}
\setlength{\headheight}{15mm} % H�he der Kopfzeile
\setheadsepline[text]{0.4pt} % Trennlinie unter Kopfzeile

% Fu�zeile
%\ifoot{} % links
\cfoot{} % mitte
\ofoot{\pagemark} % rechts

\pagestyle{scrheadings}