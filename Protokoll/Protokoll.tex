\documentclass[12pt,twoside]{article}
\usepackage[latin1]{inputenc}
\usepackage{german}
\usepackage{graphicx}
\usepackage[widemargins]{a4}
\pagestyle{empty}
\parindent0em
\parskip1.5ex plus0.5ex minus0.5ex
\topmargin4mm
\headheight5mm
\headsep6mm
\topskip5mm
\textwidth130mm
\evensidemargin40mm
\oddsidemargin5mm
\textheight259mm
\hoffset -10mm
\footskip10mm
\renewcommand{\textfraction}{0}

\begin{document}

% *********************** Titelseite ************************
\begin{center}
{\Large\sc Physik-Praktikum\\[1mm]
 
% =====> hier das Semester eintragen ------------------------
Sommersemester 2018
% -----------------------------------------------------------
}

\vspace{2cm}\hrule\vspace{1.5cm}

{\Huge\bf
Protokoll zum Versuch\\[5mm]
% =====> hier statt XXX die Versuchsnummer eintragen----<<<<<
XXX 
% -----------------------------------------------------------
\\[1cm]
{\em\Large
% =====> hier die Versuchsbezeichnung eintragen --------<<<<<
Bezeichnung des Versuchs
% -----------------------------------------------------------
}}
\end{center}
\vspace{1.5cm}\hrule\vspace{1cm}
{\large\begin{center}\begin{tabular}{p{6cm}l}
\multicolumn{2}{c}{\Large\bf Name:\ \ 
% =====> hier den eigenen Namen eintragen --------------<<<<<
Vorname Nachname
% -----------------------------------------------------------
}\\[3mm]
% -----------------------------------------------------------
\\{\em Assistent:} &
% =====> Namen des Betreuers (soweit bekannt) eintragen <<<<<
Betreuer
% -----------------------------------------------------------
\\ {\em Versuch durchgef�hrt am:} &
% ====> Datum der Versuchsdurchf"uhrung ----------------<<<<<
24.04.2018
% -----------------------------------------------------------
\end{tabular}
\end{center}
}
\newpage

\tableofcontents
\newpage

% Hier wird die Einleitung f�r das Protokoll geschrieben
\section{Einleitung}

% Hier wird die Theorie des Protokolls verfasst
\section{Theorie}
% Hier wird der Aufbau der Apparatur beschrieben, die im Versuch ben�tigt wurde
\section{Apparatur}

% Hier wird die Versuchsdurchf�hrung beschrieben
\section{Durchf�hrung der Messung}
% Hier werden die Messergebnisse ausgewertet
\section{Auswertung der Messergebnisse}

% Hier werden die Messergebnisse zusammengefasst
\section{Zusammenfassung der Messergebnisse}

% Hier werden die Messergebnisse mit anderen verglichen
\section{Diskussion}

% *********************** Beginn des Textes *****************
% empfohlene Grobgliederung:
% - Kurze Zusammenfassung der Versuchsidee
% - Theorieteil
% - Beschreibung der Apparatur
% - Auswertung mit Fehlerdiskussion
% -----------------------------------------------------------

\end{document}
